\subsection{SCRUM}
Wir haben für unser Projekt ein angepasstet SCRUM Framework verwendet:
\begin{itemize}
	\item Sprintlänge von 1-2 Wochen
	\item Sprint planning
	\item Sprint Review
	\item Management Software: Jira
	\item {\fontencoding{U}\fontfamily{futs}\selectfont\char 49\relax}Kein Daily Scrum
	\item {\fontencoding{U}\fontfamily{futs}\selectfont\char 49\relax}Keine Definition of Done
	\item {\fontencoding{U}\fontfamily{futs}\selectfont\char 49\relax}Kein Sprint Retrospective
\end{itemize}
\subsubsection{Einleitung}
In diesem Abschnitt beschreibe ich, wie ich die Rolle des Product Owners umgesetzt habe und wie diese Rolle gemäß SCRUM eigentlich vorgesehen ist.
Ich werde für jede Tätigkeit, die ich übernommen habe, erklären, wie dies nach SCRUM eigentlich umgesetzt wird.
\subsubsection{Sprintreview}
\textbf{Uni}\\
meow
\textbf{SCRUM Framework}\\
meow
\subsubsection{Sprintplanning}
\textbf{Uni}\\
Ich habe jeden Mittwoch uned auch in zwei online meetings das sprintplanning übernommen. Wir haben alle zusammen über die Storypoints(Ein Storypoint ist 8 Stunden) für die Backlogitems abgestimmt.
Wenn wir uns ungleich waren, haben wir darüber dikutiert und sind zu einem Ergebnis gekommen mit dem wir alle k;nnen
Wir haben ach dafüber  

\textbf{SCRUM Framework}\\
Als erstes wird zusammengearbeitet um ein Sprint-Ziel zu definieren
Dann werden Items aus dem Product Backlog ausgewählt mit denen das Sprintziel abgeschlossene werden kann.
Diese Items können wärenddessen verfeinert werden.
Als letztes wir jedem Items ein Zeitwert zugeordnet in dem dieses abgeschlossen werden soll.


\subsubsection{}

\subsubsection{Mögliches unterkapitel}		% wenn subsection noetig, bitte subsubsection
\begin{minted}{python}
for cake in bakery:
	eat(cake)
\end{minted}
\subsection{FP-XX anderes FP}


