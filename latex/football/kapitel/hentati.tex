\documentclass[12pt]{article}

% Sprache & Kodierung
\usepackage[utf8]{inputenc}
\usepackage[T1]{fontenc}
\usepackage[ngerman]{babel}

% Seitenlayout
\usepackage{geometry}
\geometry{a4paper, margin=2.5cm}
\usepackage{tabularx}

% Code Listings mit Minted
\usepackage{minted}
\usemintedstyle{default}
\setminted{
    fontsize=\small,
    tabsize=2,
    breaklines=true
}

% Farbige Boxen um Code
\usepackage{tcolorbox}
\usepackage{etoolbox}
\BeforeBeginEnvironment{minted}{\begin{tcolorbox}[colback=gray!5, colframe=gray!30, boxrule=0.5pt, arc=2pt]}
\AfterEndEnvironment{minted}{\end{tcolorbox}}

% Bessere Typografie
\usepackage{microtype}

% Hyperlinks
\usepackage{hyperref}

% Tiefergehende Nummerierung aktivieren
\setcounter{secnumdepth}{3}

\title{Dokumentation: Benutzerverwaltungsmodul}
\author{X, Y, 999999999\\ \and X, Z, 222222222}
\date{}

\begin{document}

\maketitle
\tableofcontents
\newpage

\section{Erfassung von Spielerstatistiken im Hudl-System}

Das System erfasst automatisch Spielerstatistiken durch Verknüpfung von \textbf{Play-Tagging} und \textbf{Spielerzuordnung}.\\
Kernfunktionen:
\begin{itemize}
    \item Rusher-Identifikation bei Laufspielzügen
    \item Passer/Receiver-Erfassung bei Passspielzügen
    \item Dynamische Anzeige von Eingabefeldern basierend auf Spieltyp
    \item Echtzeitvalidierung der Spielernummern
\end{itemize}

\subsection{Dynamische Feldanzeige (togglePlayerFields)}

\begin{minted}{javascript}
function togglePlayerFields(selectedInput) {
  const playType = selectedInput.getAttribute('data-play-type').toUpperCase();
  // Hide all fields initially
  rusher.style.display = "none";
  passer.style.display = "none";
  receiver.style.display = "none";

  // Show fields based on play type
  if (playType === "RUN") rusher.style.display = "block";
  else if (playType === "PASS" || playType === "SCREEN") {
    passer.style.display = "block";
    receiver.style.display = "block";
  }
  else if (playType === "RPO") {
    rusher.style.display = "block";
    passer.style.display = "block";
    receiver.style.display = "block";
  }
}
\end{minted}


\textbf{Trigger:} Wird bei Auswahl eines \texttt{Off Play}-Radio-Buttons aufgerufen \\
\textbf{Datenquelle:} \texttt{data-play-type} Attribut der Radio-Buttons (z.\,B. "RUN", "PASS") \\
\textbf{Logik:} Zeigt nur relevante Felder für den Spielzugtyp an

\subsection{Eingabevalidierung (validatePasser)}

\begin{minted}{javascript}
function validatePasser(input) {
  const val = parseInt(input.value);
  if (val >= 50 && val <= 79) {
    alert("Players Numbers between 50 and 79 are not accepted!");
    input.value = "";
  }
}
\end{minted}

\textbf{Prüfung:} Blockiert Nummern 50–79 (typisch für Linemen) \\
\textbf{UI-Feedback:} Alert-Nachricht + Löschung ungültiger Eingaben \\
\textbf{Anwendung:} Bei Passer/Receiver-Eingabe via \texttt{oninput}


\subsection{Initialisierungslogik}

\begin{minted}{javascript}
window.addEventListener('DOMContentLoaded', function() {
  const selectedOffPlay = document.querySelector('input[name="off_play"]:checked');
  if (selectedOffPlay) togglePlayerFields(selectedOffPlay);
});
\end{minted}

\textbf{Zweck:} Zeigt gespeicherte Felder beim Bearbeiten an \\
\textbf{Trigger:} Bei vollständig geladener Seite

\subsection{Datenmodell und Integration}

\begin{center}
\begin{tabular}{|l|l|l|}
\hline
\textbf{Feld} & \textbf{Gültige Werte} & \textbf{Abhängigkeit} \\
\hline
Rusher & 0–99 & Nur bei RUN/RPO \\
Passer & 0–49, 80–99 & Nur bei PASS/SCREEN/RPO \\
Receiver & 0–49, 80–99 & Nur bei PASS/SCREEN/RPO \\
Play Type & RUN/PASS/SCREEN/RPO & Bestimmt Sichtbarkeit \\
\hline
\end{tabular}
\end{center}

\subsection*{Herausforderungen \& Lösungen}

\begin{itemize}
    \item \textbf{Synchronisierung Frontend/Backend:}\\
    \textit{Lösung:} \texttt{data-play-type} Attribut synchronisiert Anzeigetext (value) und Spieltyp (label)

    \item \textbf{Datenkonsistenz in Tabellen:}\\
    \textit{Lösung:} Komplexe Abfragen kombinieren Positionsdaten aus verschiedenen Spielzügen

    \item \textbf{Benutzerführung:}\\
    \textit{Lösung:} Sofortiges Feedback durch Echtzeitvalidierung
\end{itemize}

\begin{minted}{html}
<!-- Beispiel: Dynamische Feldgruppen -->
<div id="rusher-field" style="display: none;">
  <input name="rusher_number" oninput="validatePasser(this)">
</div>

<!-- Beispiel: Play-Type-Bindung -->
<input type="radio" name="off_play" data-play-type="RUN" onclick="togglePlayerFields(this)">
\end{minted}

\begin{itemize}
    \item \textbf{Clean UI:} Felder sind standardmäßig versteckt (\texttt{display: none})
    \item \textbf{Responsive Validation:} \texttt{oninput} triggert Validierung bei jeder Eingabe
    \item \textbf{Kontextsensitive Logik:} \texttt{onclick} ruft Feldsteuerung nur bei Offense-Play auf
\end{itemize}

\section{Maske für Taggen von Defense}
\subsection{Purpose and Functionality}

Die Funktion zur Verteidigungs-Tagging erlaubt es Coaches, defensive Strategien während der Spielanalyse präzise zu erfassen. Wird im ODK-Selektor „Defense“ gewählt, ersetzt das System die offensiven Felder durch defense-spezifische Parameter. Dies ermöglicht:

\begin{itemize}
    \item Eine genaue Erfassung von defensiven Formationen und Taktiken
    \item Spielerbezogenes Tracking (z.\,B. Tackler, Interceptions)
    \item Nahtlose Integration mit Drive-Zusammenfassungen und Statistiken
\end{itemize}

Unabhängig vom gewählten ODK-Modus (Offense, Defense oder Special Teams) werden zentrale Spielfeldparameter standardmäßig eingeblendet. Dazu gehören:

\begin{itemize}
    \item \textbf{Down} (1–4)
    \item \textbf{Distance} (Yards bis zum First Down)
    \item \textbf{Hash} (Position: Left, Middle, Right)
    \item \textbf{Yard Line} (eigene oder gegnerische Spielfeldseite)
\end{itemize}

Diese Felder gewährleisten eine einheitliche Erfassung der Spielsituation und bilden die Grundlage für die Analyse aller Spielzüge.

Um eine korrekte Datenbasis sicherzustellen, muss vor dem Ausfüllen ein ODK-Modus gewählt werden. Sollte keine Auswahl getroffen worden sein, wird ein Warnhinweis ausgegeben, der den Benutzer auffordert, sich für Offense, Defense oder Special Teams zu entscheiden.

\subsection{Formularvalidierung für ODK-Auswahl}

Um sicherzustellen, dass ein gültiger ODK-Modus (Offense, Defense oder Special Teams) vor dem Absenden des Formulars ausgewählt wurde, wird eine JavaScript-Funktion eingesetzt. Diese prüft, ob eine der Optionen aktiviert ist. Andernfalls wird das Absenden verhindert und ein Warnhinweis angezeigt.

\begin{minted}[fontsize=\small, linenos]{javascript}
function validateForm(event) {
    const odkChoices = document.getElementsByName('odk');
    let isSelected = false;
    for (const choice of odkChoices) {
        if (choice.checked) {
            isSelected = true;
            break;
        }
    }
    if (!isSelected) {
        event.preventDefault();
        alert("Please select an ODK choice (Offense, Defense, or Special Teams).");
    }
}
\end{minted}

Diese Validierung schützt vor inkonsistenter Dateneingabe und stellt sicher, dass der Spielzug korrekt einem Bereich (Offense, Defense oder Special Teams) zugeordnet ist.

\subsection{Key Defense Parameters}

\begin{table}[H]
\centering
\begin{tabular}{|l|p{6cm}|l|}
\hline
\textbf{App Field Name} & \textbf{Beschreibung} & \textbf{Input Type} \\
\hline
Play Type          & Typ der defensiven Strategie (z. B. Base, Blitz) & Radio Buttons \\
Defense Front      & Aufstellung der D-Line (z. B. 4-3, 3-4) & Radio Buttons \\
Defense Strongside & Starke Seite der Formation (Left/Right) & Radio Buttons \\
Blitz              & Blitz-Paket-Typ & Radio Buttons \\
Slants / Stunts    & Bewegungsmuster der D-Line & Radio Buttons \\
Coverage           & Coverage-Schema der Secondary & Radio Buttons \\
Tackler 1          & Trikotnummer des Haupt-Tacklers & Number Input (0–99) \\
Tackler 2          & Trikotnummer des zweiten Tacklers & Number Input (0–99) \\
Intercepted By     & Trikotnummer des Interceptors & Number Input (0–99) \\
\hline
\end{tabular}
\caption{Defense-Parameter und Zuordnung im App-Formular}
\end{table}


\subsection{Implementation in Code}

\paragraph{Frontend (HTML/Jinja2):} Die folgende Struktur zeigt einen Ausschnitt des Defense-Formulars:

\begin{minted}[fontsize=\small,breaklines]{html}
<!-- DEFENSE SECTION (Visible when 'Defense' selected) -->
<div id="defense-section" style="display: none;">
  <!-- PLAY TYPE -->
  <div class="mb-3">
    <label class="form-label">Play type defense</label>
    <div class="btn-group d-flex flex-wrap">
      
        <input type="radio" class="btn-check" name="play_type1"
               value="{{ option.value }}"
               checked>
        <label class="btn btn-outline-primary">{{ option.value }}</label>
      
    </div>
  </div>
  ...
</div>
\end{minted}


\paragraph{Datenbank (SQLAlchemy):} Defense-spezifische Felder im Datenmodell:

\begin{minted}[fontsize=\small,breaklines]{python}
class PlayModel(db.Model):
    play_type1 = db.Column(db.String(50))
    defense_front = db.Column(db.String(50))
    defense_strongside = db.Column(db.String(50))
    blitz = db.Column(db.String(50))
    slants = db.Column(db.String(50))
    coverage = db.Column(db.String(50))
    tackler1 = db.Column(db.Integer)
    tackler2 = db.Column(db.Integer)
    interceptor = db.Column(db.Integer)
\end{minted}

\subsection{ Settings Management}

Die Verwaltung der Play-Optionen erfolgt über eine zentrale Einstellungsseite. Diese ermöglicht es dem Benutzer, sowohl für Offense als auch für Defense neue Optionen hinzuzufügen, die später im \textit{Add Play}-Formular zur Auswahl stehen.

\paragraph{Einfache Optionserweiterung}%
Es gibt einen gemeinsamen Button mit der Bezeichnung \textbf{Neue Option hinzufügen}. Nach dem Klick öffnet sich ein Formular, in dem der Benutzer:
\begin{itemize}
  \item die Kategorie auswählen kann (z.\,B. \textit{Play Type – Defense} oder \textit{Coverage – Defense}, oder entsprechende Offense-Parameter),
  \item einen Wert für die neue Option eingibt (z.\,B. \textit{Zone Blitz}),
  \item und das Formular abschickt.
\end{itemize}

Nach dem Absenden:
\begin{itemize}
  \item wird die neue Option automatisch der entsprechenden Kategorie (Offense oder Defense) zugeordnet,
  \item die Seite wird neu geladen,
  \item und die neue Option erscheint direkt im zugehörigen Abschnitt der Einstellungen, z.\,B. unter \textit{Blitz} bei den Defense-Optionen.
\end{itemize}

\paragraph{Integration mit dem Add-Play-Formular}%
Sobald eine neue Option gespeichert wurde, ist sie unmittelbar im \textit{Add Play}-Formular verfügbar, ohne dass zusätzlicher Programmieraufwand notwendig ist. Der Benutzer kann sie direkt bei der Spielzug-Eingabe auswählen.

\paragraph{Übersichtliche Darstellung}%
Alle Optionen sind in einem Accordion-Layout gruppiert und nach Parametergruppen sortiert (z.\,B. \textit{Defense Front}, \textit{Coverage}, \textit{Play Type – Offense}). Jede Option kann dort einzeln:
\begin{itemize}
  \item aktiviert oder deaktiviert werden,
  \item oder dauerhaft gelöscht werden.
\end{itemize}

\paragraph{Vorteile}%
\begin{itemize}
  \item Einfache Erweiterung der Play-Auswahl durch den Benutzer selbst
  \item Gemeinsames Interface für Offense- und Defense-Parameter
  \item Automatische Integration neuer Optionen in das Tagging-Formular
  \item Klare, gruppierte Darstellung in der Settings-Übersicht
\end{itemize}


\begin{minted}[fontsize=\small,breaklines]{html}
<!-- SETTINGS INTERFACE -->
<div id="defense-options-section">
  <div class="accordion" id="defenseOptionsAccordion">
    
      <div class="accordion-item">
        <h2 class="accordion-header">
          <button class="accordion-button">{{ display_name }}</button>
        </h2>
        <div class="accordion-body">
          <table class="table">
            ...
          </table>
        </div>
      </div>
    
  </div>
</div>
\end{minted}

\subsection{Key Features}

\begin{itemize}
  \item \textbf{Dynamisches Umschalten:} Defense-Felder ersetzen Offense-Felder bei Auswahl von „Defense“.
  \item \textbf{Validierung:} Nur gültige Nummern (0–99) werden akzeptiert.
  \item \textbf{Separate Verwaltung:} Defense-Optionen können separat konfiguriert werden.
\end{itemize}

\begin{minted}[fontsize=\small,breaklines]{javascript}
// Sichtbarkeit der Defense-Section
function showSection(section) {
  if (section === 'defense') {
    document.getElementById('defense-section').style.display = 'block';
    document.getElementById('offense-section').style.display = 'none';
  }
}

// Validierung der Spieler-Nummer
function validateInterceptedBy(input) {
  if (input.value < 0 || input.value > 99) {
    alert("Invalid jersey number (0-99)!");
    input.value = "";
  }
}
\end{minted}

\subsection{Datenfluss}

\begin{enumerate}
  \item Benutzer wählt „Defense“ im ODK-Selektor.
  \item Defense-spezifische Felder erscheinen.
  \item Daten werden validiert und gespeichert.
  \item Speicherung in \texttt{PlayModel} erfolgt:
\end{enumerate}

\begin{minted}[fontsize=\small,breaklines]{python}
play = PlayModel(
    defense_front=form.get('defense_front'),
    coverage=form.get('coverage'),
    tackler1=form.get('tackler1'),
    ...
)
\end{minted}

\subsection{Edge Cases Handling}

\begin{itemize}
  \item Leere Tackler-Felder werden als \texttt{NULL} gespeichert.
  \item Ungültige Nummern (außerhalb 0–99) werden clientseitig blockiert.
  \item Deaktivierte Optionen erscheinen nicht im Formular.
  \item Neue Formationen lassen sich jederzeit über das Einstellungsmenü hinzufügen.
\end{itemize}

\bigskip

Diese Implementierung ermöglicht eine flexible, nutzerfreundliche und strukturierte Erfassung defensiver Strategien und ist vollständig mit dem bestehenden Tagging-System kompatibel.

\section{Special Teams Tagging Feature}

\subsection{Purpose and Functionality}

Das \textit{Special Teams Tagging Feature} dient der schnellen und effizienten Erfassung von Schlüsseldaten bei Special Teams Plays. Im Gegensatz zu Offense- und Defense-Plays liegt der Fokus hier auf einer minimalistischen Oberfläche zur strukturierten Eingabe relevanter Metadaten:

\begin{itemize}
  \item Identifikation der beteiligten Spieler (Returner und Kicker)
  \item Erfassung der zurückgelegten Yards (Returner und Kicker)
  \item Klassifikation des Spielausgangs (z.\,B. Touchback, Fumble)
  \item Optimierte Oberfläche für schnelle Eingabe und geringe Fehleranfälligkeit
\end{itemize}

\subsection{Key Special Teams Parameters}

\begin{table}[H]
\centering
\begin{tabular}{|l|l|p{6cm}|l|}
\hline
\textbf{App Field Name} & \textbf{Beschreibung} & \textbf{Input Type} \\
\hline
 Returner       & Spieler, der den
Ballzurückträgt &
Number Input (0-99) \\
 Kicker         & Spieler, der den Ball kickt & Number Input (0-99) \\
Returner Yards & Zurückgelegte Yards durch Returner & Number Input (Ganzzahl) \\
 Kick Yards     & Distanz des Kicks & Number Input (Ganzzahl) \\
 Result         & Ausgang des Spielzugs & Radio Buttons (Spezifische Optionen) \\
\hline
\end{tabular}
\caption{Special Teams Parameter und Zuordnung im App-Formular}
\end{table}

\subsection{Technische Umsetzung}

\subsubsection{Frontend (HTML/Jinja2)}

Die Eingabemaske für Special Teams wird nur eingeblendet, wenn im ODK \textit{Special Teams} ausgewählt wurde. Die Felder beinhalten Returner/Kicker, Yards sowie Ergebnis-Auswahl über Radio-Buttons.

\subsubsection{JavaScript Validierung}

Spielernummern werden clientseitig geprüft, um sicherzustellen, dass nur gültige Trikotnummern (0–99) eingegeben werden:

\begin{minted}{javascript}
function validateSpecialTeam(input) {
  const value = parseInt(input.value, 10);
  if ((value >= 0 && value <= 99) || isNaN(value)) {
    input.classList.remove("is-invalid");
  } else {
    input.classList.add("is-invalid");
    alert("Player numbers must be between 0-99!");
  }
}
\end{minted}

\subsubsection{SQLAlchemy Datenmodell}

\begin{minted}{python}
class PlayModel(db.Model):
    returner = db.Column(db.Integer)        # RETURNER
    returner_yard = db.Column(db.Integer)   # RET YARDS
    kicker = db.Column(db.Integer)          # KICKER
    kicker_yard = db.Column(db.Integer)     # KICK YARDS
\end{minted}

\subsection{Key Features}

\begin{itemize}
  \item \textbf{Kontextabhängige Anzeige:} Die Special Teams-Sektion wird nur angezeigt, wenn sie ausgewählt wurde.
  \item \textbf{Echtzeitvalidierung:} Spielernummern werden sofort geprüft und hervorgehoben, wenn ungültig.
  \item \textbf{Spezialisierte Ergebnisoptionen:} Auswahlfeld für Special Teams-spezifische Ergebnisse (z.\,B. Fair Catch, Touchback).
\end{itemize}


\subsection{Edge Case Handling}

\begin{table}[h!]
\centering
\begin{tabular}{|l|p{10cm}|}
\hline
\textbf{Fall} & \textbf{Behandlung} \\
\hline
Ungültige Spielernummer & Clientseitige Validierung + Warnung \\
Fehlende Yard-Angabe & Speicherung als \texttt{NULL}, statistisch behandelt \\
Mehrere Returner & Nur ein Feld – Entscheidung liegt beim Coach \\
Unterscheidung Kick vs. Punt & Über \textit{Result}-Feld differenziert \\
\hline
\end{tabular}
\caption{Edge Case Handling für Special Teams}
\end{table}

\subsection{Business Logic}

\begin{itemize}
  \item \textbf{Nummernvalidierung:} Linemen-Nummern (50–79) können blockiert werden, wenn gewünscht.
  \item \textbf{Ergebnis-Taxonomie:} 12 spezialisierte Optionen decken alle gängigen Special Teams-Szenarien ab.
  \item \textbf{Yard-Tracking:} Nur positive Werte erlaubt (Richtung ergibt sich durch Spieltyp).
  \item \textbf{Statistische Integration:} Felder erscheinen in Drive-Summaries und Spielerstatistiken.
\end{itemize}

\begin{minted}{python}
def calculate_kicker_stats(player_id):
    kicks = PlayModel.query.filter_by(kicker=player_id).all()
    total_yards = sum(k.kicker_yard for k in kicks if k.kicker_yard)
    return {"total_kicks": len(kicks), "avg_yards": total_yards/len(kicks)}
\end{minted}

\textbf{Fazit:} Dieses Modul bietet eine schlanke, durchdachte Lösung zur Erfassung relevanter Daten bei Special Teams Plays – vollständig integriert in den bestehenden Tagging-Workflow.

\end{document}
